\documentclass[12pt, a4paper]{article}


% A pretty common set of packages
\usepackage[margin=2.5cm]{geometry}
\usepackage[T1]{fontenc}
\usepackage{graphicx}
\usepackage{amssymb}
\usepackage{amsmath}
\usepackage{color}
\usepackage{booktabs}
\usepackage{multirow}
\usepackage{engord}
\usepackage{soul}
\usepackage{textcomp}
\usepackage{parskip}
\usepackage{setspace}
\usepackage{titlesec}
\usepackage{fancyhdr}
\pagestyle{fancy}
\usepackage[UKenglish]{babel}
\usepackage[UKenglish]{isodate}
\usepackage[skip=2pt,font=footnotesize,justification=centering]{caption}
% \usepackage{natbib}
\usepackage[colorlinks=true, 
    linkcolor=blue,          % color of internal links
    citecolor=blue,        % color of links to bibliography
    filecolor=blue,      % color of file links
    urlcolor=blue]{hyperref}




% Do you prefer Sans Serif fonts?
%\usepackage{sfmath}
%\renewcommand{\familydefault}{\sfdefault} 




% Make some additional useful commands
\newcommand{\ie}{\emph{i.e.}\ }
\newcommand{\eg}{\emph{e.g.}\ }
\newcommand{\etal}{\emph{et al}}
\newcommand{\sub}[1]{$_{\textrm{#1}}$}
\newcommand{\super}[1]{$^{\textrm{#1}}$}
\newcommand{\degC}{$^{\circ}$C}
\newcommand{\wig}{$\sim$}
\newcommand{\ord}[1]{\engordnumber{#1}}
\newcommand{\num}[2]{$#1\,$#2}
\newcommand{\range}[3]{$#1$-$#2\,$#3}
\newcommand{\roughly}[2]{$\sim\!#1\,$#2}
\newcommand{\area}[3]{$#1 \! \times \! #2\,$#3}
\newcommand{\vol}[4]{$#1 \! \times \! #2 \! \times \! #3\,$#4}
\newcommand{\cube}[1]{$#1 \! \times \! #1 \! \times \! #1$}
\newcommand{\figref}[1]{Figure~\ref{#1}}
\newcommand{\eqnref}[1]{Equation~\ref{#1}}
\newcommand{\tableref}[1]{Table~\ref{#1}}
\newcommand{\secref}[1]{Section \ref{#1}}
\newcommand{\XC}{\emph{exchange-correlation}}
\newcommand{\abinit}{\emph{ab initio}}
\newcommand{\Abinit}{\emph{Ab initio}}
\newcommand{\Lonetwo}{L1$_{2}$}
\newcommand{\Dznt}{D0$_{19}$}
\newcommand{\Dtf}{D8$_{5}$}
\newcommand{\Btwo}{B$_{2}$}
\newcommand{\fcc}{\emph{fcc}}
\newcommand{\hcp}{\emph{hcp}}
\newcommand{\bcc}{\emph{bcc}}
\newcommand{\Ang}{{\AA}}
\newcommand{\inverseAng}{{\AA}$^{-1}$}
%\newcommand{\comment}[1]{}
\newcommand{\comment}[1]{\textcolor{red}{[COMMENT: #1]}}
\newcommand{\more}{\textcolor{red}{[MORE]}}
\newcommand{\red}[1]{\textcolor{red}{#1}}





% Change this to modify look of header and footer
\lhead{}
\chead{}
\rhead{}
\lfoot{}
\cfoot{\thepage{}}
\rfoot{}
\renewcommand{\headrulewidth}{0pt}
\renewcommand{\footrulewidth}{0pt}

\begin{document}

\onehalfspacing


\begin{titlepage}

\begin{center}
\includegraphics[width=1in]{figures/bham_crest}

\vspace{0.3in}

\includegraphics[width=3in]{figures/bham_logo}

\vspace{2in}

{\LARGE Transfer Learning for Alzheimer’s Disease Detection: Adapting Video Classification Models for MRI Scans }

\vspace{0.7in}

{\Large Rhys W. Alexander (2458177)}


\vfill{}
Final project report submitted\\ 
in partial fulfilment for the degree of\\
B.SCI. IN ARTIFICIAL INTELLIGENCE AND COMPUTER SCIENCE
\end{center}

\vspace{0.4in}
Date: \today{}     \hfill{} Project supervisor: \\
Word count: X,XXX   \hfill{} Dr Rickson Mesquita
\end{titlepage}






\tableofcontents

\newpage{}










\section{Abstract}
500 words

A summary of the scope and significance of the project, the methodology / techniques used, the results gained / outcomes generated and the conclusions obtained. Abstracts are generally a single paragraph and less than 250 words.





\section{Introduction}
1000 words

An introduction to the project -- your aims and objectives should go here. You can use bullet points. You probably want to keep this to half-page or so.


\section{Literature review}
2,000 words 

This should demonstrate a sound understanding of the current state of knowledge in the field, which should critically assess previous and current work in the public domain, drawing out the major facts, views, trends etc. (not just listing a series of literature / information sources and/or literature/ information abstracts). 

This should then develop into a description of the need for the project and so define the
project aims and how they are to be attained in the methodology and/or experimental work. Literature in \LaTeX{} \cite{teipel2013relevance} is best done using BibTeX \cite{gunawardena2017applying, pradhan2024analysis}. For your individual project you are asked to use the Harvard method for citations -- \LaTeX{} can do it very easily (see above).




\section{Methodology}
1500 words

Description of the methodology and/or experimental methods used to achieve aims of the project in sufficient detail to allow other researchers to repeat your work.



\section{Implementation}
1500 words


\section{Results}
1500 words

This should be a coherent presentation of the results / outcomes of your work, including your methodology and/or experimental work, such that the key findings i.e. facts, views, trends etc. are readily revealed (this should not be a chronological presentation of what was carried out nor, usually, just the raw data).

\begin{figure}
\begin{center}
\includegraphics[width=4in]{figures/bham_crest.pdf}
\caption{Some example figure}
\label{examplefig}
\end{center}
\end{figure}

A good example of how to insert a figure in \LaTeX{} is shown in \figref{examplefig}. Note that in \LaTeX{} all figures are \emph{floats}. This means they are items that flow all the way to the top of the page.








\section{Discussion}
1500 words

An analysis of the results of your methodology and/or experimental work, relating the findings back to the assessment of current state of understanding presented in the introduction.






\section{Conclusions}
1000 words

A well-defined summary of the major findings and their interpretation (this should not be further discussion of the results or speculate on areas not assessed in the results and discussion).





\bibliographystyle{IEEEtran}
\bibliography{references} 




\end{document}



