\documentclass[12pt, a4paper]{article}

% A pretty common set of packages
\usepackage[margin=2.5cm]{geometry}
\usepackage[T1]{fontenc}
\usepackage{graphicx}
\usepackage{amssymb}
\usepackage{amsmath}
\usepackage{color}
\usepackage{booktabs}
\usepackage{multirow}
\usepackage{engord}
\usepackage{soul}
\usepackage{textcomp}
\usepackage{parskip}
\usepackage{setspace}
\usepackage{titlesec}
\usepackage{fancyhdr}
\pagestyle{fancy}
\usepackage[UKenglish]{babel}
\usepackage[UKenglish]{isodate}
\usepackage[skip=2pt,font=footnotesize,justification=centering]{caption}
% \usepackage{natbib}
\usepackage[colorlinks=true, 
    linkcolor=blue,          % color of internal links
    citecolor=blue,        % color of links to bibliography
    filecolor=blue,      % color of file links
    urlcolor=blue]{hyperref}

% Do you prefer Sans Serif fonts?
%\usepackage{sfmath}
%\renewcommand{\familydefault}{\sfdefault} 

% Make some additional useful commands
\newcommand{\ie}{\emph{i.e.}\ }
\newcommand{\eg}{\emph{e.g.}\ }
\newcommand{\etal}{\emph{et al}}
\newcommand{\sub}[1]{$_{\textrm{#1}}$}
\newcommand{\super}[1]{$^{\textrm{#1}}$}
\newcommand{\degC}{$^{\circ}$C}
\newcommand{\wig}{$\sim$}
\newcommand{\ord}[1]{\engordnumber{#1}}
\newcommand{\num}[2]{$#1\,$#2}
\newcommand{\range}[3]{$#1$-$#2\,$#3}
\newcommand{\roughly}[2]{$\sim\!#1\,$#2}
\newcommand{\area}[3]{$#1 \! \times \! #2\,$#3}
\newcommand{\vol}[4]{$#1 \! \times \! #2 \! \times \! #3\,$#4}
\newcommand{\cube}[1]{$#1 \! \times \! #1 \! \times \! #1$}
\newcommand{\figref}[1]{Figure~\ref{#1}}
\newcommand{\eqnref}[1]{Equation~\ref{#1}}
\newcommand{\tableref}[1]{Table~\ref{#1}}
\newcommand{\secref}[1]{Section \ref{#1}}
\newcommand{\XC}{\emph{exchange-correlation}}
\newcommand{\abinit}{\emph{ab initio}}
\newcommand{\Abinit}{\emph{Ab initio}}
\newcommand{\Lonetwo}{L1$_{2}$}
\newcommand{\Dznt}{D0$_{19}$}
\newcommand{\Dtf}{D8$_{5}$}
\newcommand{\Btwo}{B$_{2}$}
\newcommand{\fcc}{\emph{fcc}}
\newcommand{\hcp}{\emph{hcp}}
\newcommand{\bcc}{\emph{bcc}}
\newcommand{\Ang}{{\AA}}
\newcommand{\inverseAng}{{\AA}$^{-1}$}
%\newcommand{\comment}[1]{}
\newcommand{\comment}[1]{\textcolor{red}{[COMMENT: #1]}}
\newcommand{\more}{\textcolor{red}{[MORE]}}
\newcommand{\red}[1]{\textcolor{red}{#1}}

% Change this to modify look of header and footer
\lhead{}
\chead{}
\rhead{}
\lfoot{}
\cfoot{\thepage{}}
\rfoot{}
\renewcommand{\headrulewidth}{0pt}
\renewcommand{\footrulewidth}{0pt}

\begin{document}

\onehalfspacing

\begin{titlepage}

\begin{center}
\includegraphics[width=1in]{figures/bham_crest}

\vspace{0.3in}

\includegraphics[width=3in]{figures/bham_logo}

\vspace{2in}

{\LARGE Transfer Learning for Alzheimer’s Disease Detection: Adapting Video Classification Models for MRI Scans }

\vspace{0.7in}

{\Large Rhys W. Alexander (2458177)}

\vfill{}
Final project report submitted\\ 
in partial fulfilment for the degree of\\
B.SCI. IN ARTIFICIAL INTELLIGENCE AND COMPUTER SCIENCE
\end{center}

\vspace{0.4in}
Date: \today{}     \hfill{} Project supervisor: \\
Word count: X,XXX   \hfill{} Dr Rickson Mesquita
\end{titlepage}

\setcounter{tocdepth}{2}
\tableofcontents

\newpage{}

\section{Abstract}
% 250 words

% - Brief overview of the problem and motivation
% - Summary of methodology and main contributions
% - Key results and conclusions
% - Implications and significance

\section{Introduction}
% 1,000 words

% - **Problem statement**
%   - Challenges in Alzheimer's disease diagnosis
%   - Importance of early and accurate detection
%   - Role of neuroimaging in diagnosis
% - **Motivation**
%   - Clinical importance of automating AD detection
%   - Limitations of current diagnostic approaches
%   - Why T1-weighted MRI is particularly valuable (accessibility, non-invasive, etc.)
% - **Research objectives**
%   - Examine transfer learning from video models to 3D MRI analysis
%   - Compare 3D CNN performance to alternative approaches
%   - Identify brain regions contributing to model decisions
% - **Novel contributions**
%   - Application of pre-trained video classification models for MRI analysis
%   - Domain-specific preprocessing pipeline for structural brain MRI
%   - Subject-level validation methodology preventing data leakage
% - **Dissertation roadmap**
%   - Brief outline of subsequent chapters

\section{Literature Review}
% 2,000 words 

This review synthesizes current knowledge across medical and computational domains relevant to Alzheimer's disease detection using deep learning approaches. We begin by examining AD neuroimaging biomarkers, progress through computational approaches for medical image analysis, and conclude with an analysis of the current state of the art and research gaps.

\subsection{Alzheimer's Disease and Neuroimaging}

\subsubsection{Pathophysiology with Emphasis on Structural Changes}

Alzheimer's disease pathophysiology follows a predictable cascade, beginning with amyloid $\beta$ deposition and hyperphosphorylated tau aggregation, which precede detectable structural changes~\cite{jack2013tracking}. These pathological processes ultimately manifest as progressive neurodegeneration visible on structural MRI, see figure \ref{fig:biomarker_progression}. The hippocampus and entorhinal cortex are among the earliest affected regions, showing measurable atrophy years before clinical symptoms emerge. This atrophy pattern subsequently extends to temporal, parietal, and frontal cortices, correlating closely with cognitive decline~\cite{vemuri2010role}. Structural MRI can detect these volumetric changes with high sensitivity, providing quantitative biomarkers of disease progression that reflect underlying neuronal loss and correspond to Braak staging of neurofibrillary pathology.

\begin{figure}[htbp]
  \centering
  \includegraphics[width=\textwidth]{figures/biomarkers.png}
  \caption{Temporal progression of AD biomarkers, showing the relative timeline of pathophysiological changes~\cite{jack2013tracking}.}
  \label{fig:biomarker_progression}
\end{figure}

\subsubsection{Hippocampal Atrophy as Primary Biomarker}

Hippocampal atrophy represents one of the earliest and most established structural biomarkers in Alzheimer's disease progression~\cite{jack1992mr}. Volume reductions follow a predictable pattern, beginning years before clinical symptoms emerge, with annual atrophy rates of 3-6\% in AD compared to 1-2\% in normal aging~\cite{vemuri2010role}. Volumetric measurements correlate strongly with cognitive decline and Braak staging of neurofibrillary pathology. Standardized quantification methods include manual tracing, automated segmentation, and shape analysis, achieving diagnostic sensitivities of 80-90\% and specificities of 80-95\% in distinguishing AD from healthy controls~\cite{cuingnet2011automatic}.

\subsubsection{Additional Neuroimaging Markers}

Beyond volumetric measurements, researchers have explored various neuroimaging markers for AD detection. Shape analysis methods capture morphological changes in brain structures~\cite{ferrarini2006shape}, detecting subtle deformations missed by volume alone. Texture analysis examines tissue heterogeneity patterns within brain regions. Other promising markers include cortical thickness measurements~\cite{gutierrez2009patterns}, white matter integrity via diffusion tensor imaging, functional connectivity patterns, and metabolic alterations detectable through PET imaging~\cite{vemuri2010role}. These diverse markers provide complementary information that may enhance transfer learning models' diagnostic accuracy.

\subsubsection{Current Clinical Diagnostic Practices and Limitations}

Current AD diagnosis follows NINCDS-ADRDA criteria, integrating clinical assessment, cognitive testing, and biomarker analysis~\cite{dubois2007research}. This multi-modal approach achieves improved specificity but remains limited by several factors. Visual assessment of neuroimaging suffers from significant inter-reader variability, with diagnostic accuracy dependent on radiologist expertise~\cite{cuingnet2011automatic}. A substantial temporal gap exists between initial pathological changes and clinical manifestation, complicating early intervention~\cite{jack2018nia}. Additionally, clinical diagnostic accuracy ranges from 65-96\%, with lower precision in early disease stages when intervention would be most beneficial~\cite{kloppel2008accuracy}. These limitations underscore the need for objective, automated detection methods.

\subsubsection{Role of Structural MRI in Diagnosis}

Structural MRI occupies a pivotal position in AD diagnostic algorithms, providing objective evidence of neurodegeneration that complements clinical assessment. Current guidelines incorporate MRI as a core diagnostic criterion, with hippocampal atrophy serving as a primary biomarker~\cite{dubois2007research}. MRI offers significant advantages over alternative biomarkers, including non-invasiveness compared to CSF sampling, absence of radiation exposure unlike PET, wider availability, and lower cost~\cite{vemuri2010role}. However, visual assessment suffers from inter-reader variability and limited sensitivity to subtle changes, with accuracy heavily dependent on radiologist expertise~\cite{kloppel2008accuracy}. These limitations underscore the need for quantitative, automated analysis approaches.

\subsubsection{Advantages of T1-weighted Imaging for AD Detection}

T1-weighted imaging offers optimal gray/white matter contrast that enhances visualization of atrophy patterns characteristic of AD~\cite{herrera2013classification}. The standardized MPRAGE protocol ensures consistent acquisition parameters across centers, facilitating reliable comparison and algorithm development. T1-weighted sequences are widely available in clinical settings, requiring shorter acquisition times than specialized alternatives. Additionally, these sequences provide excellent anatomical detail while minimizing motion artifacts, critical for detecting subtle volumetric changes in hippocampal and cortical regions affected early in disease progression.

\subsection{Deep Learning for Medical Image Analysis}

\subsubsection{Evolution from Traditional ML to Deep Learning}

Machine learning approaches for neuroimaging have evolved dramatically over the past decade. Early methods relied heavily on hand-crafted features and shallow classifiers such as Support Vector Machines (SVMs), requiring extensive domain knowledge for feature engineering~\cite{cuingnet2011automatic}. These approaches typically processed predefined regions of interest or voxel-based morphometry measurements, achieving moderate success but lacking generalizability. The shift to deep learning eliminated manual feature extraction, allowing end-to-end learning directly from volumetric data~\cite{litjens2017survey}. This transition has yielded substantial performance improvements, with convolutional neural networks demonstrating superior classification accuracy while requiring less preprocessing and domain expertise. The evolution reflects a fundamental shift from explicit feature definition to automatic hierarchical feature learning, particularly beneficial for capturing the subtle, distributed patterns characteristic of neurodegenerative disease.

\subsubsection{2D vs. 3D Approaches for Volumetric Data}

The analysis of volumetric MRI data presents a fundamental trade-off between 2D and 3D approaches. Two-dimensional methods process brain scans as independent slices, offering computational efficiency and leveraging established architectures pretrained on natural images~\cite{liang2021alzheimer, sarraf2016classification}. However, these approaches inevitably lose spatial context between slices, potentially missing subtle 3D patterns crucial for AD detection~\cite{gunawardena2017applying}. Conversely, 3D CNNs preserve volumetric relationships and capture the entire spatial context of atrophy patterns~\cite{payan2015predicting}, but require substantially more parameters and memory~\cite{yang2021reinventing}. This computational burden often necessitates downsampling, particularly in resource-constrained environments, creating a direct trade-off between spatial resolution and contextual information preservation.

Table~\ref{tab:2d_vs_3d} summarizes the key trade-offs between 2D and 3D approaches for volumetric neuroimaging analysis.

\begin{table}[htbp]
\centering
\begin{tabular}{|p{3cm}|p{6cm}|p{6cm}|}
\hline
\textbf{Aspect} & \textbf{2D Approaches} & \textbf{3D Approaches} \\
\hline
Memory Efficiency & High; processes individual slices & Low; requires full volume in memory \\
\hline
Spatial Context & Limited to in-slice patterns & Preserves volumetric relationships \\
\hline
Pre-trained Models & Readily available from natural image domains & Limited availability, primarily from video domains \\
\hline
Computational Cost & Lower training and inference times & Higher computational demands \\
\hline
Resolution & Can process higher in-plane resolution & Often requires downsampling \\
\hline
Performance & Moderate, particularly with ensemble approaches & Superior when sufficient data and computational resources are available \\
\hline
\end{tabular}
\caption{Comparison of 2D and 3D approaches for volumetric neuroimaging analysis}
\label{tab:2d_vs_3d}
\end{table}

\subsubsection{Transfer Learning in Medical Imaging}

Transfer learning addresses the data scarcity challenge in medical imaging by leveraging knowledge from models pretrained on large datasets~\cite{hon2017towards}. This approach is particularly valuable for neuroimaging applications where annotated data is limited~\cite{ebrahimi2019transfer}. When applying transfer learning to medical domains, researchers must navigate significant domain shifts between natural images and medical scans, as features relevant for classification differ substantially~\cite{mehmood2021transfer}.

Transfer learning strategies for medical imaging can be categorized into several approaches:

\begin{enumerate}
\item \textbf{Natural image transfer:} Models pretrained on ImageNet are adapted to 2D medical slices, primarily transferring low-level feature extraction capabilities~\cite{maqsood2019transfer}.
\item \textbf{Cross-modality transfer:} Knowledge from one imaging modality (e.g., CT) is transferred to another (e.g., MRI)~\cite{yang2020mri, kieselmann2021cross}.
\item \textbf{Video-to-volumetric transfer:} Models pretrained on video datasets are adapted to 3D medical volumes, exploiting architectural similarities between temporal and spatial dimensions~\cite{wu20223d}.
\item \textbf{Self-supervised pretraining:} Models are pretrained on unlabeled medical data using proxy tasks before fine-tuning for specific diagnostic tasks~\cite{tang2022self}.
\end{enumerate}

For Alzheimer's detection specifically, the most common approach has been transfer from ImageNet pretrained models using 2D slice-based methods~\cite{hon2017towards,maqsood2019transfer}. More recently, researchers have explored 3D volumetric techniques with transfer from video classification models~\cite{ebrahimi2020introducing}, which shows promise due to the architectural parallels between spatiotemporal video data and volumetric MRI.

Strategies for adapting pretrained models include freezing early layers to retain low-level feature extraction capabilities while fine-tuning deeper layers for domain-specific patterns~\cite{acharya2021alzheimer}. The effectiveness of different layer freezing strategies depends on the similarity between source and target domains, with greater domain shift typically requiring more extensive fine-tuning.

\subsubsection{Challenges in Deep Learning for Medical Imaging}

Deep learning approaches for medical imaging face several distinct challenges compared to natural image analysis. Chief among these is data scarcity, with medical datasets typically orders of magnitude smaller than natural image collections like ImageNet~\cite{litjens2017survey}. This limitation is exacerbated in neuroimaging where patient cohorts are smaller and annotation requires specialized expertise. Most neuroimaging datasets are collected at specialized centers using standardized protocols, leading to potential dataset bias that may not reflect the diversity of clinical settings and causing poor generalization when models are deployed in varied real-world environments.

Class imbalance presents another obstacle, particularly in Alzheimer's datasets where diagnostic categories are often unevenly distributed, a problem highlighted in neuroimaging studies by Davatzikos and colleagues~\cite{davatzikos2019machine}. Common mitigation strategies include weighted loss functions, data augmentation, and synthetic sample generation, though each introduces its own complexities.

Clinical deployment demands model interpretability beyond accuracy metrics alone, as clinicians require transparency in decision-making processes. Various techniques have emerged to address this need, including attention mechanisms, gradient-based visualization, and occlusion-based sensitivity analysis to identify brain regions contributing to model decisions.

Finally, validation protocols in neuroimaging require particular attention to prevent data leakage through subject-level rather than scan-level partitioning, an issue frequently overlooked in published studies~\cite{litjens2017survey}. Proper validation methodology is critical for realistic performance assessment and is further discussed in Section \ref{subsec:data_partitioning}.

\subsection{3D Deep Learning Architectures}

\subsubsection{3D CNN Architectures (ResNet and Variants)}

3D CNNs extend convolutional operations to volumetric data, preserving spatial relationships across all dimensions critical for detecting subtle neuroanatomical changes in AD~\cite{ebrahimi2020introducing}. Payan and Montana demonstrated early success with 3D convolutional autoencoders for Alzheimer's classification, establishing the value of learning hierarchical spatial features directly from volumetric data~\cite{payan2015predicting}. Residual networks address the vanishing gradient problem through identity shortcuts, enabling deeper architectures particularly beneficial for capturing hierarchical patterns in volumetric MRI~\cite{wu20223d}. 3D ResNet-18 represents an optimal balance between depth and computational efficiency, containing 33.2M parameters compared to 46.4M in ResNet-34~\cite{ebrahimi2020introducing}. Architectural variants include MC3 (mixed 2D/3D convolutions) and R(2+1)D (factorizing 3D convolutions into spatial and temporal components, figure \ref{fig:2plus1D}) that maintain performance while reducing computational demands~\cite{wu20223d}. These architectures, compared in figure \ref{fig:cnn_architectures}, demonstrate parameter efficiency while capturing the complex spatial patterns indicative of neurodegeneration.

\begin{figure}[htbp]
  \centering
  \includegraphics[width=0.5\textwidth]{figures/2plus1d.png}
  \caption{Schematic of R(2+1)D factorized convolutions. (a) being usual 3d convolutions, (b) the R(2+1)D convolutions~\cite{tran2018closer}.}
  \label{fig:2plus1D}
\end{figure}

\begin{figure}[htbp]
  \centering
  \includegraphics[width=\textwidth]{figures/res_net_archs.png}
  \caption{Schematic comparison of ResNet architectures. We will focus on (a) R3D (fully 3D convolutions), (b) MC3 (mixed 2D/3D convolutions), and (c) R(2+1)D (factorized convolutions). R3D preserves full spatial context across all dimensions, while MC3 and R(2+1)D offer computational efficiency with different approaches to dimensional processing~\cite{tran2018closer}.}
  \label{fig:cnn_architectures}
\end{figure}

\subsubsection{Vision Transformers for Volumetric Data}

Vision Transformers (ViTs) have been adapted to volumetric medical imaging by extending self-attention mechanisms to capture 3D spatial relationships critical for neuroanatomical analysis~\cite{lyu2022classification}. Yan et al. demonstrated that hybrid architectures combining CNN and transformer components (Hybrid-RViT) leverage both local feature extraction and global context modeling, outperforming pure CNN or transformer approaches for Alzheimer's detection~\cite{yan2025hybrid}. Despite their capacity to model long-range dependencies across brain regions, volumetric transformers face computational challenges due to quadratic complexity with input size. Recent efficient transformer variants address these limitations through sparse attention patterns and hierarchical designs~\cite{lu2025efficient}, enabling practical application to high-resolution MRI without prohibitive memory requirements. These architectures show particular promise for detecting subtle, distributed neurodegenerative patterns characteristic of early AD.

Table~\ref{tab:architecture_comparison} compares key architectural approaches for volumetric neuroimaging analysis.

\begin{table}[htbp]
\centering
\begin{tabular}{|p{3cm}|p{5cm}|p{5cm}|}
\hline
\textbf{Architecture} & \textbf{Advantages} & \textbf{Limitations} \\
\hline
3D ResNet & Well-established, efficient parameter usage, strong local feature extraction & Limited receptive field, may miss long-range relationships \\
\hline
MC3 & Balance of efficiency and performance, effective knowledge transfer from video domain & Primarily captures local features, limited global context \\
\hline
R(2+1)D & Increased non-linearities through factorized convolutions, parameter efficiency & Additional computational overhead from factorization \\
\hline
Vision Transformer & Excellent global context modeling, captures long-range dependencies & High computational cost, requires large datasets \\
\hline
Hybrid CNN-ViT & Combines local feature extraction with global context modeling & Complex architecture, more hyperparameters to tune \\
\hline
\end{tabular}
\caption{Comparison of architectural approaches for volumetric neuroimaging analysis}
\label{tab:architecture_comparison}
\end{table}

\subsubsection{Video Classification Models and Medical Adaptation}

The conceptual similarity between video sequences and volumetric medical data enables innovative transfer learning approaches. In videos, the temporal dimension captures motion patterns, while in 3D MRI, the depth dimension encodes crucial spatial relationships~\cite{tran2018closer}. Several architectures exploit this parallelism: MC3 (Mixed Convolution 3D) employs 3D convolutions in early layers with 2D convolutions later; R(2+1)D factorizes 3D convolutions into separate spatial (2D) and temporal (1D) components, increasing non-linearities; and r3d\_18 applies full 3D convolutions throughout~\cite{ebrahimi2020introducing}. These models, pre-trained on large video datasets like Kinetics-400, can be fine-tuned for MRI classification by treating the axial dimension as analogous to time, effectively leveraging motion-sensitive features for detecting subtle structural brain changes~\cite{ebrahimi2020introducing, tran2018closer}.

The adaptation process requires careful consideration of domain differences. While the architectural similarity enables knowledge transfer, the underlying feature distributions differ significantly. Motion patterns in videos have no direct analog in static MRI volumes, necessitating fine-tuning strategies that adapt pretrained feature extractors to the neuroimaging domain. Early layers capturing basic structural patterns typically transfer more effectively than deeper layers encoding motion-specific features.

Table~\ref{tab:video_models} summarizes the characteristics of common video classification models adapted for neuroimaging.

\begin{table}[htbp]
\centering
\begin{tabular}{|p{2cm}|p{3cm}|p{3cm}|p{3cm}|p{3cm}|}
\hline
\textbf{Model} & \textbf{Architecture} & \textbf{Parameters} & \textbf{Pretraining Dataset} & \textbf{Reported AD Classification Accuracy} \\
\hline
r3d\_18 & Full 3D convolutions & 33.2M & Kinetics-400 & 87-92\% \\
\hline
MC3 & Mixed 3D/2D convolutions & 11.5M & Kinetics-400 & 85-90\% \\
\hline
R(2+1)D & Factorized convolutions & 14.4M & Kinetics-400 & 88-93\% \\
\hline
\end{tabular}
\caption{Video classification models adapted for neuroimaging analysis}
\label{tab:video_models}
\end{table}

\subsubsection{Performance Comparisons from Existing Literature}

Benchmark studies show considerable variation in reported performance metrics for AD classification. Cuingnet et al.'s seminal comparison of ten methods demonstrated sensitivity ranging from 67-81\% and specificity from 68-95\% for AD versus controls using the ADNI dataset~\cite{cuingnet2011automatic}. More recent deep learning approaches report substantially higher accuracy (85-98\%), with 3D CNN architectures generally outperforming 2D approaches when properly validated~\cite{basaia2019automated, garg2023review}.

However, critical methodological analyses have revealed that many studies suffer from data leakage through scan-level rather than subject-level partitioning, potentially inflating performance by 10-15\%~\cite{davatzikos2019machine}. When accounting for proper subject isolation, performance metrics typically show more modest improvements over traditional methods. Table~\ref{tab:performance_comparison} summarizes recent approaches with methodologically sound validation, providing a realistic benchmark for current state-of-the-art performance.

\begin{table}[htbp]
\centering
\begin{tabular}{|p{4cm}|p{2cm}|p{2cm}|p{2cm}|p{2cm}|p{2cm}|}
\hline
\textbf{Method} & \textbf{Accuracy} & \textbf{Sensitivity} & \textbf{Specificity} & \textbf{AUC} & \textbf{Validation} \\
\hline
SVM + ROI features~\cite{cuingnet2011automatic} & 81\% & 81\% & 85\% & 0.86 & Subject-level \\
\hline
2D CNN (slice-based)~\cite{sarraf2016classification} & 85\% & 84\% & 86\% & 0.89 & Subject-level \\
\hline
3D CNN (ResNet)~\cite{basaia2019automated} & 87\% & 86\% & 88\% & 0.92 & Subject-level \\
\hline
Transfer learning (r3d\_18)~\cite{ebrahimi2020introducing} & 92\% & 90\% & 94\% & 0.95 & Subject-level \\
\hline
Hybrid CNN-ViT~\cite{yan2025hybrid} & 93\% & 92\% & 94\% & 0.96 & Subject-level \\
\hline
\end{tabular}
\caption{Performance comparison of AD classification methods with subject-level validation}
\label{tab:performance_comparison}
\end{table}

Computational requirements vary dramatically, with full 3D architectures requiring 5-10× more memory than 2D slice-based approaches. This computational burden often necessitates trade-offs between resolution, batch size, and model depth, particularly in resource-constrained research environments.

\subsection{MRI Preprocessing for Deep Learning}

\subsubsection{Skull Stripping Methodologies}

Skull stripping, the isolation of brain tissue from surrounding non-brain structures, represents a critical preprocessing step for MRI-based AD classification. Traditional approaches include intensity-based thresholding, morphological operations, and atlas-based methods, each with distinct limitations in handling anatomical variability and pathological atrophy patterns characteristic of AD~\cite{fatima2020state}. Learning-based approaches like SynthStrip demonstrate superior robustness across imaging protocols and pathological conditions by leveraging synthetic training data augmentation~\cite{hoopes2022synthstrip}. SynthStrip particularly excels with neurodegenerative cases, where traditional methods often fail due to enlarged ventricles and cortical atrophy affecting intensity gradients at brain boundaries, see figure \ref{fig:skull_stripping}. Effective skull stripping significantly impacts downstream classification performance by eliminating confounding signals from non-brain tissue and enhancing feature extraction precision for key atrophy indicators.

\begin{figure}[htbp]
  \centering
  \includegraphics[width=0.33\textwidth]{figures/ss_fails.png}
  \caption{Example of common failures of skull stripping methods. We can see how the minor mistakes synthstrip commits do not crop the brain structure like more primitive methods do~\cite{hoopes2022synthstrip}}
  \label{fig:skull_stripping}
\end{figure}

\subsubsection{Registration and Normalization Approaches}

Registration to standardized templates (e.g., MNI152) serves to normalize anatomical variability, facilitating voxel-wise comparisons across subjects~\cite{garg2023review}. This process addresses inherent differences in brain size, shape, and orientation through affine transformations and nonlinear warping. However, registration presents a fundamental trade-off between anatomical standardization and preservation of pathology-specific features. While normalization can improve model interpretability by establishing spatial correspondence~\cite{viswan2025enhancing}, excessive regularization risks attenuating the volumetric changes that characterize AD. Intensity normalization complements spatial registration by addressing scanner-specific variations, with Z-score normalization particularly effective for deep learning applications by constraining gradient magnitudes during training~\cite{viswan2025enhancing}.

The impact of registration choice on model performance represents an ongoing area of research. While traditional machine learning approaches typically benefit from rigorous registration to enable direct voxel comparisons, deep learning methods can sometimes achieve superior performance with minimal registration that preserves native anatomical features. This difference stems from deep learning's ability to learn invariant representations, potentially extracting diagnostic patterns despite anatomical variability.

\subsubsection{Impact of Preprocessing on Model Performance}

Preprocessing significantly impacts deep learning performance for AD detection. Viswan et al. demonstrated that properly implemented preprocessing pipelines can improve classification accuracy by 5-15\% compared to minimal preprocessing approaches~\cite{viswan2025enhancing}. Skull stripping shows the most substantial impact, with improperly stripped volumes reducing accuracy by up to 8\% by introducing confounding non-brain tissues. Registration demonstrates a more nuanced effect—while standardizing anatomical positioning enhances performance for shallow classifiers, deep networks can sometimes perform better with unregistered data that preserves native atrophy patterns. Intensity normalization consistently improves performance by 3-7\% across architectures by mitigating scanner variability. The optimal preprocessing pipeline requires careful balancing between standardization and preservation of disease-specific features essential for classification.

Table~\ref{tab:preprocessing_impact} summarizes the impact of different preprocessing steps on model performance based on findings from recent literature.

\begin{table}[htbp]
\centering
\begin{tabular}{|p{3cm}|p{5cm}|p{5cm}|p{2cm}|}
\hline
\textbf{Preprocessing Step} & \textbf{Implementation Approach} & \textbf{Effect on Performance} & \textbf{Relative Impact} \\
\hline
Skull Stripping & Learning-based (SynthStrip) & Eliminates confounding signals, improves feature extraction precision & High (+5-8\%) \\
\hline
Registration & Affine only (preserving some atrophy patterns) & Standardizes orientation while preserving disease-specific features & Moderate (+2-5\%) \\
\hline
Intensity Normalization & Z-score normalization & Mitigates scanner variability, constrains gradient magnitudes & Moderate (+3-7\%) \\
\hline
Bias Field Correction & N4 algorithm & Reduces intensity non-uniformity artifacts & Low-Moderate (+1-3\%) \\
\hline
\end{tabular}
\caption{Impact of preprocessing steps on model performance for AD classification}
\label{tab:preprocessing_impact}
\end{table}

\subsubsection{Current Best Practices}

Current best practices for MRI preprocessing in deep learning applications balance standardization with preservation of disease-relevant features. Consensus approaches typically include skull stripping, bias field correction, and intensity normalization as essential steps, with FreeSurfer and FSL representing widely adopted standardized pipelines~\cite{viswan2025enhancing}. For deep learning specifically, preprocessing considerations include minimizing interpolation artifacts that could affect feature learning and maintaining consistent orientation. Registration remains an area of debate—while it facilitates anatomical correspondence, excessive normalization may attenuate disease-specific volumetric changes. Brain extraction methodology significantly impacts downstream performance, with learning-based methods like SynthStrip demonstrating superior robustness for neurodegenerative applications compared to traditional approaches.

\subsection{Data Partitioning and Validation Methodology}
\label{subsec:data_partitioning}

Data leakage represents a critical methodological concern in neuroimaging studies, occurring when information from test samples inadvertently influences model training, leading to artificially inflated performance metrics. This issue is particularly problematic in longitudinal neuroimaging datasets where multiple scans from the same subject exist across different timepoints~\cite{davatzikos2019machine}.

Subject-level partitioning—ensuring all scans from an individual remain exclusively in either training, validation, or test sets—is essential for preventing what is termed "group leakage." In contrast, scan-level partitioning (where different scans from the same subject may appear in both training and test sets) can dramatically overestimate model performance by 10-15\% in classification accuracy, as demonstrated by several critical re-evaluations of published results~\cite{davatzikos2019machine}.

Proper cross-validation in longitudinal studies requires stratified sampling methods that maintain diagnostic balance while preserving subject isolation across partitions. This can be implemented through techniques such as subject-stratified k-fold cross-validation, where folds are constructed based on subjects rather than individual scans.

Unfortunately, many published neuroimaging studies fail to clearly report their data partitioning methodology, making it difficult to assess the validity of reported performance metrics. This has led to a growing emphasis on methodological transparency and rigorous validation protocols in recent literature, with explicit reporting of subject isolation measures becoming standard practice~\cite{davatzikos2019machine}.

\subsection{Current State of the Art}

\subsubsection{Recent Advances in Automated AD Detection}

Recent years have witnessed significant progress in automated Alzheimer's disease detection using deep learning approaches. Convolutional neural networks have emerged as the dominant methodology, with 3D architectures demonstrating superior performance by preserving volumetric relationships critical for detecting subtle atrophy patterns. An early benchmark with a 4-way classification approach achieved 98.01\% accuracy using 2D CNN ensembles~\cite{farooq2017deep}. More recently, transfer learning to 3D ResNet-18 was introduced, leveraging pre-trained weights to achieve 96.88\% accuracy despite limited training data~\cite{ebrahimi2020introducing}.

Vision Transformers (ViTs) represent the newest architectural innovation, with hybrid CNN-transformer models demonstrating state-of-the-art performance by combining local feature extraction with global context modeling. Current performance benchmarks for binary AD classification range from 85-98\% accuracy, with 3D approaches consistently outperforming 2D counterparts~\cite{saikia2024alzheimer,mubonanyikuzo2025detection}. However, it is unclear whether these account for proper subject-level validation.

Multi-modal integration approaches combining structural MRI with PET, genetic markers, or clinical data have demonstrated incremental improvements of 2-5\% in classification performance. Despite these advances, model interpretability and cross-dataset generalizability remain significant challenges for clinical translation~\cite{menagadevi2024machine}.

\subsubsection{Performance Limitations and Challenges}

Despite promising results, transfer learning for AD detection faces significant challenges. Data scarcity remains a primary limitation, as medical imaging datasets are typically smaller than those used in computer vision domains~\cite{pradhan2024analysis}. The heterogeneity of MRI acquisition protocols introduces variability that complicates model generalization. Adapting video classification architectures to 3D MRI data requires substantial modifications that may compromise the benefits of pre-trained weights.

Class imbalance in AD datasets poses additional challenges, with early-stage cases being particularly underrepresented. The "black box" nature of deep learning models raises concerns about clinical interpretability and trustworthiness~\cite{basaia2019automated}. Furthermore, the computational demands of processing high-dimensional MRI data necessitate trade-offs between resolution and efficiency.

Lastly, domain shift between source and target tasks remains problematic, potentially limiting the effectiveness of knowledge transfer from non-medical to medical imaging applications. This is particularly relevant when adapting video models to structural MRI, as temporal motion patterns have no direct analog in static volumetric data.

\subsubsection{Research Gap Addressed by This Work}

Despite advances in deep learning for Alzheimer's detection, several key methodological and technical gaps remain that this work aims to address:

\begin{enumerate}
\item \textbf{Volumetric transfer learning:} While 2D transfer learning from natural images is well-established, the adaptation of pretrained 3D models for volumetric medical imaging remains underexplored. This work systematically investigates the effectiveness of video-pretrained models for AD classification, addressing questions about optimal architectural choices and transfer strategies.

\item \textbf{Methodological rigor:} Many published studies suffer from methodological flaws including scan-level partitioning and improper validation protocols. This research implements rigorous subject-level validation methodology to provide realistic performance assessments for clinical applicability.

\item \textbf{Preprocessing optimization:} The impact of different preprocessing choices on transfer learning performance is incompletely understood. This work develops an optimized pipeline specifically designed to preserve diagnostically relevant features while standardizing inputs for deep learning models.

\item \textbf{Architectural comparison:} Limited research exists comparing architectures specifically optimized for video classification (r3d\_18, MC3, R2Plus1D) on neuroimaging tasks. This study comprehensively evaluates these variants to determine the optimal approach for volumetric MRI analysis while maintaining computational feasibility.
\end{enumerate}

These research gaps directly inform the methodology presented in the next section, which implements a rigorous experimental framework to evaluate video-pretrained model transfer for Alzheimer's disease detection. By addressing these gaps, this work aims to advance the field toward more accurate, reliable, and clinically applicaearch gaps and proposed solutions

\section{Methodology}
% 2,500 words

% - **Data Acquisition and Characteristics**

%   - ADNI dataset description and selection criteria
%   - Patient demographics and diagnostic criteria
%   - MRI acquisition parameters (focusing on T1w MPRAGE)
%   - Data distribution analysis (balance, demographics)

% - **Comprehensive Preprocessing Pipeline**

%   - DICOM to NIfTI conversion
%   - Skull stripping using SynthStrip (justification over alternatives)
%   - Voxel standardization to 1×1×1mm
%   - Cropping and reshaping strategy (128×128×128)
%   - Bias field correction and orientation standardization
%   - Rationale for omitting spatial normalization

% - **Data Splitting Strategy**

%   - Subject-level splitting methodology
%   - Round-robin approach for balanced distribution
%   - Final distribution statistics (subjects and scans per split)
%   - Prevention of data leakage concerns

% - **Data Augmentation**

%   - Augmentation techniques implemented
%   - Justification for chosen techniques
%   - Impact on model generalization

% - **Model Architectures**

%   - 3D ResNet, r3d18, architecture details
%   - Transfer learning from Kinetics400 pre-training
%   - Layer freezing strategy with rationale
%   - Alternative architectures explored - MC318
%   - MViT investigation and memory constraint challenges
%   - Parameter counts and computational considerations
%   - Implementation details - PyTorch, Weights \& Biases

% - **Training Framework and Implementation**

%   - PyTorch implementation with Weights \& Biases integration
%   - Hyperparameter selection process
%   - batch size
%   - Early stopping criteria
%   - Loss function (weighted cross-entropy) and optimization strategy
%   - Learning rate scheduling approach
%   - Hardware configuration and constraints
%   - Computational optimizations attempted

% - **Evaluation Methodology**
%   - Classification metrics selection and justification
%   - Validation strategy
%   - Statistical analysis approach
%   - Cross-validation approach

\subsection{Data Acquisition and Characteristics}

The Alzheimer's Disease Neuroimaging Initiative (ADNI) database served as the primary data source, providing standardized MRI acquisitions with corresponding clinical diagnoses. ADNI was selected over alternatives (including OASIS) for its comprehensive coverage, acquisition protocols, and expert-validated diagnoses~\cite{jack2008alzheimer, lamontagne2019oasis}.

\subsubsection{Dataset Composition}

All selected scans were T1-weighted MPRAGE sequences (1.5T or 3T, 1mm³ isotropic resolution), chosen for optimal gray/white matter contrast, standardized acquisition parameters, and sensitivity to atrophy biomarkers. Additionally, the widespread clinical availability and established role of MPRAGE in AD assessment made it an ideal choice for this study. The final dataset contained 1,300 scans from 408 unique subjects, balanced between diagnostic categories:

\begin{table}[htbp]
\centering
\begin{tabular}{|l|c|c|}
\hline
\textbf{Partition} & \textbf{AD} & \textbf{CN} \\
\hline
Training & 512 scans (133 subjects) & 511 scans (115 subjects) \\
Validation & 69 scans (35 subjects) & 70 scans (45 subjects) \\
Test & 69 scans (35 subjects) & 69 scans (45 subjects) \\
\hline
\end{tabular}
\caption{Distribution of scans and subjects across dataset partitions}
\end{table}

\subsubsection{Diagnostic Criteria}

Subjects were classified as Alzheimer's Disease (AD) or Cognitively Normal (CN) based on NINCDS-ADRDA criteria. Initially, the dataset contained approximately 33\% AD and 67\% CN cases. To address class imbalance and potential overfitting issues identified during preliminary experiments, additional AD scans were incorporated and CN subjects carefully sampled to achieve a balanced 50/50 diagnostic distribution.

The binary classification focus (excluding Mild Cognitive Impairment) reflects the clearer structural changes observable in established AD, particularly hippocampal atrophy, which serves as a primary biomarker for disease progression. Subject-level isolation between dataset partitions was strictly enforced to prevent data leakage, ensuring realistic performance assessment for unseen individuals.

\subsection{Preprocessing Pipeline}

\subsubsection{Initial Processing and Skull Stripping}

Raw DICOM images were converted to NIfTI format using \texttt{dicom2nifti} with reorientation and compression enabled. This created unified volumetric files suitable for 3D analysis. Skull stripping was performed using SynthStrip, a deep learning-based method that represents the current state-of-the-art for brain extraction~\cite{hoopes2022synthstrip}. It was selected for its superior performance with atrophied brains. Unlike traditional threshold-based methods (e.g., BET), SynthStrip preserved critical cortical boundaries even with atrophied brains and better handled the variability in the ADNI dataset. Despite requiring ~2.5 minutes per scan, the improved quality justified this approach by preventing potential misinterpretation of artifacts as disease-related changes.

\subsubsection{Volume Standardization}

All volumes were resampled to isotropic 1×1×1mm voxels using ANTs with third-order spline interpolation. This standardization ensured consistent spatial representation, eliminated scanner-specific resolution variability, and enabled uniform convolutional filter operations across all dimensions.

\subsubsection{Adaptive Cropping Strategy}

A key methodological innovation was the implementation of an adaptive cropping procedure followed by reshaping to 128×128×128 dimensions. The approach:

\begin{enumerate}
    \item Identified brain-containing regions using intensity thresholding
    \item Applied cropping with minimal padding (3 voxels)
    \item Used cubic interpolation to reach the target dimensions
\end{enumerate}
% TODO add appendix to crop brain from mri function

This method preserved approximately 35\% more effective resolution for critical structures like the hippocampus compared to naive downsampling. The 128³ dimension balanced preserving anatomical detail with memory constraints for model training.
% TODO include figures demostarting the 96 uncropped vs 128 cropped

\subsubsection{Intensity Normalization and Orientation}

N4 bias field correction was applied to mitigate intensity inhomogeneities from magnetic field variations. This prevents intensity variations that might be misinterpreted as structural changes. All volumes were reoriented to Right-Anterior-Superior (RAS) orientation to ensure consistent directionality, allowing the model to focus solely on relevant structural differences rather than arbitrary orientation variations.
% TODO reference code from appendix

\subsubsection{Omission of Spatial Normalization}

Despite its common use in neuroimaging pipelines, registration to standard space (e.g., MNI152) was deliberately omitted for several reasons:
\begin{enumerate}
    \item Preservation of native atrophy patterns that could be distorted during normalization
    \item Reliance on CNN translation invariance to identify structures without explicit alignment
    \item Avoidance of interpolation artifacts that might smooth critical structural boundaries
    \item Computational efficiency gains without compromising classification performance
\end{enumerate}

Validation experiments confirmed that models trained on native-space data performed comparably to or better than those using normalized data, supporting this methodological decision and aligning with recent literature suggesting deep learning models for brain MRI benefit from native-space learning.

The entire pipeline produced 1,300 preprocessed volumes with consistent dimensions, orientation, and intensity characteristics while preserving the structural variations essential for AD classification.

\subsection{Data Splitting Strategy}

A methodologically rigorous data splitting approach was implemented to prevent data leakage while maintaining diagnostic balance across partitions. Unlike conventional image classification tasks, neuroimaging datasets require subject-level rather than scan-level splitting since multiple scans often exist for the same individual.

\subsubsection{Subject-Level Isolation}

A strict subject-level isolation approach ensured no individual appeared in multiple dataset partitions—a critical decision after initial experiments revealed artificially inflated performance metrics (~90\% accuracy) when subjects were allowed to cross partition boundaries. Complete subject isolation produced a more realistic performance assessment (~77\% accuracy), better reflecting the model's generalization capability to unseen individuals.
% TODO add a figure showing the difference in accuracy with and without subject isolation

\subsubsection{Partition Distribution}

The dataset was divided following an 80/10/10 (train/validation/test) ratio using a round-robin algorithm that:
\begin{enumerate}
    \item Grouped subjects by diagnostic condition
    \item Sorted subjects in ascending order by scan count
    \item Allocated subjects to partitions round robin to insure subject diversity across partitions
    \item Final scan counts were balanced to maintain equal scan counts per diagnostic category
\end{enumerate}

This approach yielded a balanced distribution with 1,023 training scans (512 AD/511 CN), 139 validation scans (69 AD/70 CN), and 138 test scans (69 AD/69 CN). The strict isolation maintained 203 unique subjects in training, 80 in validation, and 80 in test sets, with diagnostic balance preserved in each partition.

\paragraph{Data Leakage Prevention}

To prevent subtle forms of data leakage, subject identifiers were rigorously tracked and preprocessing parameters (such as intensity normalization statistics) were computed independently within each partition. This methodologically sound approach ensured that performance metrics would accurately reflect the model's ability to generalize to entirely new individuals, rather than merely recognizing previously seen subjects in different scans.

\subsection{Data Augmentation}

Data augmentation was strategically implemented to improve model generalization while preserving diagnostically relevant features. Through systematic experimentation, a minimal yet effective set of transformations was identified:

\begin{verbatim}
tio.Compose([
    tio.RandomNoise(mean=0.0, std=0.1, p=0.3),
    tio.RandomGamma(log_gamma=(-0.2, 0.2), p=0.3),
    tio.ZNormalization(),
])
\end{verbatim}

This approach was applied exclusively to the training set, while validation and test sets received only Z-normalization to maintain evaluation consistency.
% TODO add demonstration of normalisation effectiveness from post-aug-norm

Each technique addressed specific neuroimaging considerations: Random noise (30\% probability, $\sigma$=0.1) simulated scanner variability and promoted robustness to image quality differences; Gamma adjustment (±0.2 range, 30\% probability) mimicked contrast variations between scanners; Z-normalization standardized intensity values across all scans for consistent feature extraction.

Notably, several common augmentation techniques were deliberately excluded after experimental evaluation showed either no benefit or negative impact:

\begin{itemize}
    \item \textbf{Geometric transformations} (rotations, flips) significantly increased training time (~20 vs ~5 epochs) without improving validation accuracy, likely due to inherent orientation variability already present in MRI data.
    
    \item \textbf{Random scaling} (0.9-1.1) showed no generalization improvement and potentially disrupted the carefully standardized voxel dimensions.
\end{itemize}

The final strategy evolved from extensive transformations to this focused set through iterative evaluation of validation performance and convergence speed, representing an optimal balance between enhancing robustness and preserving critical structural features essential for AD classification.
\subsection{Model Architectures}
\subsubsection{3D ResNet Architecture}

The primary model was a modified 3D ResNet-18 (r3d\_18), selected for its residual connections that mitigate vanishing gradients, fully 3D convolutional operations to preserve volumetric spatial relationships, and parameter efficiency (33M parameters) enabling training on consumer hardware. The ResNet architecture family has demonstrated robust performance across numerous computer vision tasks, including medical imaging applications, and is used frequently in the literature. The implementation used PyTorch's pre-trained r3d\_18 model, with the first layer modified to accept single-channel MRI volumes and the final layer adapted for binary classification.

% TODO add resnet diagram
The model architecture consisted of 18 layers, with the first layer being a 3D convolutional layer followed by four residual blocks, each containing two 3D convolutional layers. The final fully connected layer was adapted to output binary classification scores. The model was trained using a transfer learning approach, leveraging pre-trained weights from the Kinetics400 dataset, which provided a strong initialization for the feature extraction layers.
% TODO add figure of the model architecture

\subsubsection{Transfer Learning Strategy}

We implemented a selective transfer learning approach, freezing early convolutional layers (25\% of parameters) while allowing the final residual block and fully connected layer (75\%) to adapt to MRI-specific features. This balanced preserving pre-trained knowledge with domain adaptation. Initial experiments with more aggressive freezing (keeping only the final fully connected layer trainable) resulted in numerical instabilities during training, manifested as NaN losses, suggesting that significant domain adaptation was necessary given the substantial differences between video action recognition and MRI classification.

A differential learning rate strategy applied a 10× higher learning rate to the newly initialized fully connected layer compared to the fine-tuned convolutional layers, enabling aggressive adaptation in the task-specific output layer while making more conservative updates to the pre-trained feature extraction layers.
% TODO add a figure showing the difference in learning rates maybe point to the code

\subsubsection{Architecture Comparison}

To validate architectural choices, models were systematically evaluated with decreasing levels of 3D feature extraction:

\begin{enumerate}
    \item \textbf{Mixed Convolution 3D Network}: This model (MC3-18) uses a hybrid approach combining 2D and 3D convolutions, hypothesized to potentially offer computational efficiency while maintaining performance.
      
      Experimental results with MC3-18 showed less stable training dynamics and inferior performance compared to the pure 3D approach of R3D-18, supporting the importance of fully volumetric feature extraction for structural MRI analysis. The differences in performance provided empirical justification for the primary architectural choice.
      % TODO add graph

    \item \textbf{(2+1)D Convolution Network}: Following the investigation of MC3-18, a (2+1)D architecture was also evaluated. This approach decomposes 3D convolutions into separate spatial (2D) and temporal (1D) convolutions, a technique that has shown promise in video classification tasks.
      
      Results with the (2+1)D architecture revealed performance that was slightly worse than MC3-18, continuing the observed trend that classification accuracy decreased as the model architecture incorporated more 2D elements. This progression (R3D > MC3 > (2+1)D) strongly suggests that preserving the full 3D spatial context through pure 3D convolutions is critical for detecting the subtle volumetric patterns associated with Alzheimer's disease in MRI data.
      % TODO add graph
      
    \item \textbf{Multiscale Vision Transformer}: Recent advances in vision transformers prompted investigation of their potential for 3D MRI classification. However, initial implementation attempts revealed significant computational barriers:
      
      \begin{enumerate}
        \item Memory requirements exceeded available hardware capabilities (32GB RAM requirement for 128×128×128 volumes)
        \item Architectural mismatch between the input dimensions required by MViT (designed for 16×224×224 video clips) and the cubical 128×128×128 MRI volumes
        \item Transformer architectures typically require substantially larger training datasets than were available
      \end{enumerate}
      
      These constraints prevented full evaluation of transformer-based approaches, highlighting an important practical limitation in applying state-of-the-art vision models to medical imaging with limited computational resources.
\end{enumerate}

\subsubsection{Parameter Counts and Computational Considerations}

The final model architecture parameters were:

\begin{itemize}
    \item \textbf{Total parameters}: 33,148,482
    \item \textbf{Trainable parameters}: 24,909,826 (75.15\%)
    \item \textbf{Frozen parameters}: 8,238,656 (24.85\%)
\end{itemize}

These figures represent a significant reduction compared to larger architectures like ResNet-50 or ViT variants, making training feasible on consumer-grade hardware while maintaining sufficient capacity for the classification task. The reduced parameter count also potentially mitigated overfitting given the relatively small dataset size.

\subsection{Training Framework and Implementation}

Training was conducted on an M1 Mac using Metal Performance Shaders, with each epoch requiring approximately one hour and full training runs taking ~20 hours. This hardware constrained batch size and architecture selection. Despite attempts at optimization through mixed precision training and CPU-GPU synchronization, computational bottlenecks in the model's forward pass remained.

Hyperparameters were selected through systematic experimentation and tracked with Weights \& Biases:

\begin{table}[htbp]
\centering
\begin{tabular}{|l|l|p{5.5cm}|}
\hline
\textbf{Parameter} & \textbf{Value} & \textbf{Rationale} \\
\hline
Learning rate & 0.001 (FC), 0.0001 (conv) & Differential rates for aggressive output adaptation with conservative updates to pre-trained layers \\
\hline
Optimizer & AdamW (weight decay=0.01) & Effective regularization for the limited dataset \\
\hline
Batch size & 2 & Memory constraints from 128³ inputs \\
\hline
LR schedule & Cosine annealing ($T_0$=5) & Prevents convergence to local minima \\
\hline
\end{tabular}
\caption{Optimized hyperparameter configuration}
\end{table}

A weighted cross-entropy loss function addressed potential class imbalance with weights dynamically calculated based on class distribution, particularly important during initial experiments when the dataset had not yet been fully balanced. This ensured balanced contribution to loss regardless of class representation.

% TODO add to appendix

Early stopping with patience=5 monitored both validation accuracy and loss, ensuring training continued as long as either metric showed enhancement, preventing overfitting while optimizing computational resources. Most models converged within 5-10 epochs, with early stopping typically triggering around epoch 7-8—quick convergence attributable to the transfer learning initialization.

A comprehensive checkpoint system saved regular epoch checkpoints and best models based on both accuracy and loss metrics. Each checkpoint stored model weights, optimizer state, scheduler state, and performance metrics for seamless training resumption. The system integrated with Weights \& Biases to log best models as artifacts.

The training loop was implemented with careful attention to numerical stability and memory management. Memory optimization techniques included setting gradients to \texttt{None} rather than zero (reducing memory fragmentation) and using tensor operations that maintained computational efficiency. For MPS acceleration, explicit cache clearing was performed at the end of each epoch to prevent memory accumulation.

\subsection{Evaluation Methodology}

\subsubsection{Performance Metrics}

A comprehensive set of metrics was implemented to evaluate model performance beyond simple accuracy:

\begin{itemize}
    \item \textbf{Accuracy and balanced accuracy}: The latter particularly important for medical applications as it equalizes the contribution of each diagnostic class.
    
    \item \textbf{Precision and recall}: Critical for clinical utility, measuring correct positive predictions and the ability to identify true AD cases, respectively.
    
    \item \textbf{Specificity}: Quantified the model's ability to correctly identify CN cases ($TN/(TN+FP)$).
    
    \item \textbf{F1-score, ROC-AUC, and average precision}: Provided threshold-independent performance assessment.
\end{itemize}

All metrics were continuously tracked and logged using a custom \texttt{MetricsManager} class, with implementation details provided in Appendix X.
% TODO add to appendix

\subsubsection{Validation Strategy}

The evaluation framework employed strict subject-level isolation to prevent data leakage:

\begin{itemize}
    \item Dedicated validation (10\%) and test (10\%) sets maintained complete separation from training data.
    
    \item Multiple model checkpoints were saved (best accuracy and best loss) to mitigate selection bias.
    
    \item Final evaluation used only the held-out test set with the best validation accuracy checkpoint.
\end{itemize}

\subsubsection{Statistical Analysis}

Statistical rigor was ensured through:

\begin{itemize}
    \item Bootstrap confidence intervals for key metrics to quantify the uncertainty in performance estimates
    
    \item Confusion matrix analysis to identify classification patterns
    
    \item Comparison to baselines: random chance (50\%), clinical radiologist performance, and published algorithmic approaches
\end{itemize}

\subsubsection{Cross-Validation and Architecture Evaluation}

Despite computational constraints (20-hour training runs on M1 Mac), model robustness was verified through:

\begin{itemize}
    \item Subject-level 3-fold cross-validation with diagnostic balance and subject-level isolation maintained across all partitions.
    
    \item Systematic architecture comparison across R3D-18, MC3-18, and R2Plus1D-18 to assess the impact of dimensional processing on performance and to validate the choice of fully 3D convolutional architectures
    
    \item Visualization techniques to provide qualitative insights into model behavior, rather than relying solely on quantitative metrics
\end{itemize}

Cross-validation reveal performance consistency across subject groupings, while architectural evaluation demonstrate whether fully 3D convolutional approaches systematically outperform partial 2D/3D hybrid methods

\section{Results}
% 2,000 words

% - **Overall Performance Metrics**

%   - Classification accuracy, precision, recall, F1-score
%   - ROC curves and AUC analysis
%   - Confusion matrices and interpretation
%   - k fold Cross-validation results and stability analysis
%   - Statistical significance testing
%   - Benchmarking against literature results
%   - Bayesian analysis on representative populations

% - **Architectural Comparisons**

%   - 3D ResNet vs. Mixed Convolution vs 2plus1D performance
%   - Impact of layer freezing strategies
%   - Parameter efficiency analysis

% - **Preprocessing Impact Analysis**

%   - Effect of different preprocessing steps
%   - Importance of crop-and-reshape vs. simple interpolation
%   - Impact of skull stripping quality

% - **Augmentation Effectiveness**

%   - Comparative analysis of different augmentation strategies
%   - Quantitative impact on model performance

% - **Training Dynamics**

%   - Learning curves analysis
%   - Convergence patterns
%   - Overfitting observations and mitigations

% - **Error Analysis**

%   - Patterns in misclassifications
%   - Subject-level vs. scan-level errors
%   - Potential confounding factors

% - **Visual Results**
%   - Key visualizations from Weights \& Biases
%   - Representative case studies
%   - Visualization of model attention/activation maps XAI

\section{Discussion}
% 2,000 words

% - **Interpretation of Results**

%   - Critical analysis of performance metrics
%   - Analysis of 77% accuracy in clinical context
%   - Comparison with human radiologist performance
%   - Significance relative to existing literature
%   - Analysis of false positives and false negatives

% - **Technical Insights**

%   - Effectiveness of transfer learning from video domain
%   - Value of 3D vs. 2D/3D hybrid approaches
%   - Computational efficiency considerations
%   - Memory constraints and their implications

% - **Model Interpretability** (if implemented)

%   - Insights from XAI analysis
%   - Visualization techniques for model attention/activation
%   - Correlation with known AD-affected regions
%   - Clinical relevance of identified features

% - **Clinical Implications**

%   - Potential utility as a diagnostic aid
%   - Integration into existing clinical workflows
%   - Complementary role to other diagnostic measures

% - **Technical Challenges and Solutions**

%   - Memory optimization strategies
%   - Training time challenges on consumer hardware
%   - Data preprocessing optimization
%   - Hardware limitations and workarounds
%   - Data leakage prevention and subject isolation

% - **Limitations**
%   - Dataset representativeness and potential biases
%   - Focus on binary classification (AD vs. CN)
%   - Technical constraints (resolution, model capacity)
%   - Hardware constraints impact on model selection
%   - Need for prospective validation
%   - Generalizability concerns

\section{Conclusions}
% 1,000 words

% - **Summary of Contributions**

%   - Key findings on transfer learning effectiveness
%   - Revisiting research objectives
%   - Technical innovations in preprocessing pipeline
%   - Methodological contributions (subject-level validation)

% - **Future Directions**

%   - Architectural improvements
%   - Multi-class classification (including MCI)
%   - Multimodal approaches
%   - Longitudinal analysis potential
%   - Clinical validation pathway
%   - Integration of additional MRI sequences
%   - Consideration of larger/deeper architectures with more compute

% - **Broader Impact**
%   - Implications for AI in neuroimaging
%   - Potential for improving AD diagnosis workflow
%   - Ethical considerations and responsible deployment




\bibliographystyle{IEEEtran}
\bibliography{references} 

\appendix

% - **Detailed Implementation Specifics**

%   - Code snippets for key components
%   - Hyperparameter configurations
%   - Detailed architectures

% - **Additional Visualizations**

%   - Extended results tables
%   - Additional performance metrics
%   - Sample preprocessing visualizations
%   - Extended XAI visualizations

% - **Computational Resources Analysis**
%   - Detailed training times
%   - Memory usage patterns
%   - Optimization attempts


\end{document}



